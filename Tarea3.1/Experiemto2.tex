\documentclass[acmlarge]{acmart}

\usepackage{booktabs} % For formal tables
\usepackage{graphicx}
\usepackage[spanish]{babel}


\usepackage[ruled]{algorithm2e} % For algorithms
\renewcommand{\algorithmcfname}{ALGORITHM}
\SetAlFnt{\small}
\SetAlCapFnt{\small}
\SetAlCapNameFnt{\small}
\SetAlCapHSkip{0pt}
\IncMargin{-\parindent}

% Metadata Information
\acmJournal{PACMHCI}
\acmVolume{1}
\acmNumber{1}
\acmArticle{1}
\acmYear{2025}
\acmMonth{12}
\acmArticleSeq{1}

%\acmBadgeR[http://ctuning.org/ae/ppopp2016.html]{ae-logo}
%\acmBadgeL[http://ctuning.org/ae/ppopp2016.html]{ae-logo}


% Copyright
\setcopyright{none}
%\setcopyright{acmlicensed}
%\setcopyright{rightsretained}
%\setcopyright{usgov}
%\setcopyright{usgovmixed}
%\setcopyright{cagov}
%\setcopyright{cagovmixed}

% DOI
%\acmDOI{0000001.0000001}


% Document starts
\begin{document}
	% Title portion
	\title{Experimento 2} 
	\author{GÓMEZ EDGAR}
	\affiliation{%
		\institution{Universidad Central de Venezuela}
		\city{Caracas}
		\country{Venezuela}}
	
	\begin{abstract}
	Este experimento implementa la creación y gestión de múltiples procesos hijos desde un proceso padre en Python. Se demuestra la capacidad de generar y sincronizar cinco procesos concurrentes utilizando \texttt{subprocess.Popen} y \texttt{wait()}. Se analiza el comportamiento de concurrencia, la asignación de PID únicos y la coordinación entre procesos, conceptos esenciales para el desarrollo de sistemas distribuidos y aplicaciones paralelas.
	\end{abstract}
	
	\ccsdesc[500]{Organización de sistemas informáticos~Sistemas Distribuidos}
	\ccsdesc[300]{Organización de sistemas informáticos~Hilos en software}
	\ccsdesc[300]{Organización de sistemas informáticos~Procesos en software}
	
	\keywords{Sistemas Distribuidos, Arquitectura de Software, procesos en software, hilos en software}
	
	\maketitle
	
	\section{Introducción}
	La gestión eficiente de múltiples procesos concurrentes es fundamental en sistemas distribuidos y aplicaciones modernas que requieren paralelismo. Este experimento explora la creación y coordinación de varios procesos hijos desde un único proceso padre en Python. A diferencia del modelo padre-hijo simple, este enfoque permite analizar patrones de concurrencia real, donde múltiples tareas se ejecutan simultáneamente mientras se mantiene la capacidad de sincronización. Se examinan conceptos como la asignación de PID, la ejecución no determinista en entornos concurrentes y la importancia de mecanismos de espera para garantizar la finalización ordenada de procesos.
	
	\section{Proceso Padre}
	
	El proceso padre realiza los siguientes pasos secuencialmente, muestra el PID del proceso padre, luego se inician cinco procesos hijos, posteriormente se muestran los PID de los porcesos hijos, se espera que termine los preoceos hijos y luego se muestra un mensaje de que ha finalizado el proceso padre.
	
	\begin{verbatim}
		import os
		import subprocess
		
		print(f"Hola, soy el proceso padre. Mi PID es: {os.getpid()}")
		print("Creando procesos hijos...")
		
		# Iniciar el proceso hijo
		# 'subprocess.Popen' permite que el padre continúe su ejecución
		array = []
		for i in range(5):
			new_proceso_hijo = subprocess.Popen(["python", "hijo.py"]) 
			array.append(new_proceso_hijo)
		
		for proceso_hijo in array:
			print(f"El proceso hijo se ha creado con el PID: {proceso_hijo.pid}")
		
		# El proceso padre espera a que el hijo termine
		# Esto demuestra la relación padre-hijo
		for proceso_hijo in array:
			proceso_hijo.wait()
			print(f"El proceso hijo con PID {proceso_hijo.pid} ha terminado.")
		
		print("Adiós, soy el proceso padre.")
	\end{verbatim}
	\section{Proceso Hijo}
	
	El procesos hijo muestra su PID, espera cinco segundos y luego muestra un mensaje de que finalizado la tarea.
	
	\begin{verbatim}
		import os
		import time
		
		print(f"Hola, soy el proceso hijo. Mi PID es: {os.getpid()}")
		time.sleep(5) # Simula una tarea que toma 5 segundos
		print("El proceso hijo ha terminado su tarea.")
	\end{verbatim}
	
	\section{Resultados de la ejecución}
	
	Ejecuanto el código mostrado de los procesos obtenemos en la terminal la siguiente respuesta, como se observa en la Fig 1.
	
	\begin{figure}[H]
		\centering
		\includegraphics[width=0.7\linewidth]{multipleHijos}
		\caption[Fig 1]{resultado de ejecutar multiples procesos hijos}
		\label{fig:multiplehijos}
	\end{figure}
	
	Se pueden observar los PID de cada proceso que de muestra que son procesos completamente diferentes, se muestra la creacion de cinco procesos hijos con PID únicos, tambien se puede observar que la ejecución es concurente y que no existe un orden en que los mensajes son mostrados, se observa que al final el proceso padre espera la terminación de todos los procesos hijos. 
	
	\section{Conclusión}
	El experimento demostró exitosamente la creación y gestión de múltiples procesos hijos concurrentes. Se observó que cada proceso recibe un PID único y que la ejecución presenta características no deterministas, donde el orden de los mensajes varía entre ejecuciones. El uso de \texttt{wait()} permitió al padre sincronizar la finalización de todos los hijos, asegurando una terminación ordenada. Estos resultados destacan la importancia de los mecanismos de coordinación en aplicaciones paralelas y distribuidas, donde la gestión adecuada de múltiples procesos es crucial para el rendimiento y la confiabilidad del sistema.
	
\end{document}
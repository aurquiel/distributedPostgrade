\documentclass[acmlarge]{acmart}

\usepackage{booktabs} % For formal tables
\usepackage{graphicx}
\usepackage[spanish]{babel}
\usepackage{amsmath}
\usepackage{algorithmic}

\usepackage[ruled]{algorithm2e} % For algorithms
\renewcommand{\algorithmcfname}{ALGORITHM}
\SetAlFnt{\small}
\SetAlCapFnt{\small}
\SetAlCapNameFnt{\small}
\SetAlCapHSkip{0pt}
\IncMargin{-\parindent}

% Metadata Information
\acmJournal{PACMHCI}
\acmVolume{1}
\acmNumber{1}
\acmArticle{1}
\acmYear{2026}
\acmMonth{1}
\acmArticleSeq{1}

% Copyright
\setcopyright{none}

% Document starts
\begin{document}
	
	% Title portion
	\title{Practica de Python problema de Generales Bizantinos} 
	\author{GÓMEZ EDGAR}
	\affiliation{%
		\institution{Universidad Central de Venezuela}
		\city{Caracas}
		\country{Venezuela}}
	
	\begin{abstract}
  Este trabajo presenta una implementación práctica en Python del \textit{Problema de los Generales Bizantinos},se investiga experimentalmente la condición teórica $N \geq 3m + 1$, necesaria para alcanzar un consenso a pesar de la presencia de $m$ traidores. Mediante la ejecución de un script de simulación con diferentes configuraciones de generales leales (l) y traidores (t), se demuestra cómo el incumplimiento de esta condición conduce a resultados inconsistentes, mientras que su cumplimiento permite alcanzar un acuerdo mayoritario.
	\end{abstract}
	
	\ccsdesc[500]{Organización de sistemas informáticos~Sistemas Distribuidos}
	\ccsdesc[300]{Organización de sistemas informáticos~Tolerancia a Fallas}
	
	\keywords{Sistemas Distribuidos, Tolerancia a Fallos, Generales Bizantinos, Consenso Distribuido}
	
	\maketitle
	
	\section{Introducción}
	El Problema de los Generales Bizantinos modela el desafío de lograr consenso en un sistema distribuido donde algunos componentes pueden fallar o actuar de manera maliciosa, siendo fundamental para la tolerancia a fallos en redes y sistemas críticos. Esta práctica implementa en Python una simulación del algoritmo clásico para validar experimentalmente la condición teórica $N \geq 3m + 1$, donde $N$ es el número total de generales y $m$ el número de traidores tolerables. A través de diferentes configuraciones de nodos leales y traidores, se analiza la capacidad del sistema para alcanzar un acuerdo unánime a pesar de la información contradictoria, demostrando la relación directa entre la configuración del sistema y la posibilidad de consenso.
	
	\section{Resultados}
	
	\subsection{Primer Comando}
	
	Ejecutamos en consola el comando y se obtiene el resultado lo que se muestra en Fig 1.
	
	\begin{figure}[H]
		\centering
		\includegraphics[width=0.7\linewidth]{salida1}
		\caption{}
		\label{fig:salida1}
	\end{figure}
	
	
	Analizando la salida, se observa que no se alcanza un consenso entre los generales. Los generales 0 y 2 muestran un empate en su decisión (un voto para 'ATTACK' y uno para 'RETREAT'), lo que indica una falta de acuerdo unánime. Este resultado corrobora que no se cumple la relación $N \geq 3m + 1$. En este caso, con tres generales ($N=3$) y un traidor ($m=1$), se tiene $3 \geq 4$, lo cual es falso.
	
	\subsection{Segundo Comando}
	
	Ejecutamos en consola el comando y se obtiene el resultado lo que se muestra en Fig 2.
	
	\begin{figure}[H]
		\centering
		\includegraphics[width=0.7\linewidth]{salida2}
		\caption{}
		\label{fig:salida2}
	\end{figure}
	
	
	 En este escenario, nuevamente no se alcanza un consenso. Todos los generales (0, 1 y 2) presentan vectores de decisión con un voto para cada opción, evidenciando la indecisión provocada por el traidor. Esto vuelve a demostrar el incumplimiento de la condición $N \geq 3m + 1$ ($3 \geq 4$ es falso), imposibilitando un acuerdo claro.  
	
	\subsection{Tercer Comando}
	
	Ejecutamos en consola el comando y se obtiene el resultado lo que se muestra en Fig 3.
	
	\begin{figure}[H]
		\centering
		\includegraphics[width=0.7\linewidth]{salida3}
		\caption{}
		\label{fig:salida3}
	\end{figure}
	
	
	En esta configuración, se observa un claro consenso hacia la acción 'ATTACK'. Los generales 0, 2 y 4 muestran una mayoría de tres votos para atacar frente a uno para retirarse, mientras que los generales 1 y 3 registran cuatro votos para atacar. Este acuerdo mayoritario confirma que se cumple la relación $N \geq 3m + 1$. Con cinco generales ($N=5$) y un traidor ($m=1$), se tiene $5 \geq 4$, lo cual es cierto, permitiendo al sistema superar la influencia del traidor y alcanzar una decisión coherente.
	
	\section{Conclusión}
	La implementación práctica en Python del algoritmo de los Generales Bizantinos ha permitido validar experimentalmente la condición teórica $N \geq 3m + 1$ para alcanzar consenso en presencia de fallos arbitrarios. Los resultados demuestran que cuando esta condición no se cumple (tres generales con un traidor), el sistema genera decisiones inconsistentes e incompatibles, mientras que al satisfacerla (cinco generales con un traidor) se logra un acuerdo mayoritario claro y coherente. Esta práctica confirma la importancia crítica de la relación entre el número total de nodos y la cantidad de fallos tolerables en el diseño de sistemas distribuidos resilientes, ilustrando cómo la teoría se materializa en resultados computacionales concretos.	

\end{document}
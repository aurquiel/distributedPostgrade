\documentclass[acmlarge]{acmart}

\usepackage{booktabs} % For formal tables
\usepackage{graphicx}
\usepackage[spanish]{babel}


\usepackage[ruled]{algorithm2e} % For algorithms
\renewcommand{\algorithmcfname}{ALGORITHM}
\SetAlFnt{\small}
\SetAlCapFnt{\small}
\SetAlCapNameFnt{\small}
\SetAlCapHSkip{0pt}
\IncMargin{-\parindent}

% Metadata Information
\acmJournal{PACMHCI}
\acmVolume{1}
\acmNumber{1}
\acmArticle{1}
\acmYear{2026}
\acmMonth{1}
\acmArticleSeq{1}

%\acmBadgeR[http://ctuning.org/ae/ppopp2016.html]{ae-logo}
%\acmBadgeL[http://ctuning.org/ae/ppopp2016.html]{ae-logo}


% Copyright
\setcopyright{none}
%\setcopyright{acmlicensed}
%\setcopyright{rightsretained}
%\setcopyright{usgov}
%\setcopyright{usgovmixed}
%\setcopyright{cagov}
%\setcopyright{cagovmixed}

% DOI
%\acmDOI{0000001.0000001}


% Document starts
\begin{document}
	% Title portion
	\title{Relojes de Lamport Procesos} 
	\author{GOMEZ EDGAR}
	\affiliation{%
		\institution{Universidad Central de Venezuela}
		\city{Caracas}
		\country{Venezuela}}
	
	
	\begin{abstract}
		Los relojes lógicos de Lamport constituyen un mecanismo fundamental para ordenar eventos en sistemas distribuidos asíncronos sin depender de una sincronización física precisa. Este trabajo presenta el algoritmo, basado en contadores que se incrementan localmente y se ajustan al intercambiar mensajes, y lo ilustra mediante un ejercicio práctico con tres procesos. La solución detallada demuestra cómo se asigna un orden causal consistente a los eventos, validando la utilidad del esquema como base para la coordinación en entornos distribuidos.
	\end{abstract}
	
	
	%
	% The code below should be generated by the tool at
	% http://dl.acm.org/ccs.cfm
	% Please copy and paste the code instead of the example below. 
	%
	
	\ccsdesc[500]{Organización de sistemas informáticos~Sistemas Distribuidos}
	\ccsdesc[300]{Organización de sistemas informáticos~Relojes de Lamport}
	
	%
	% End generated code
	%
	
	\keywords{Sistemas Distribuidos, Asincronía}
	
	\maketitle
	
	\section{Introduction}
	
	Los relojes o tiempos lógicos de Lamport, propuestos por Leslie Lamport en 1978, son un mecanismo utilizado en sistemas distribuidos para ordenar eventos causalmente sin requerir sincronización física exacta. Utilizan contadores que se incrementan en cada evento y se actualizan al recibir mensajes, estableciendo una relación de sucesos.
	
	El algoritmo sigue las siguientes reglas:
	
	- Un proceso incrementa su contador antes de cada evento que ocurra en ese proceso.
	- Cuando un proceso envía un mensaje, este incluye su contador en el envío.
	- Al recibir un mensaje, se actualiza el contador del receptor si es necesario, al mayor entre su propio contador y la marca de tiempo recibida en dicho mensaje.
	
	\begin{verbatim}
	\end{verbatim}
	
	En pseudocódigo para proceso interno
	
	\begin{verbatim}
		reloj = reloj + 1;
		marca_temporal_mensaje = reloj;
	\end{verbatim}
 	
	En pseudocódigo el algoritmo para enviar un mensaje es:
	
	
	\begin{verbatim}
		reloj = reloj + 1;
		marca_temporal_mensaje = reloj;
		enviar(mensaje, marca_temporal_mensaje);
	\end{verbatim}
	
	Algoritmo para la recepción del mensaje:
	
	\begin{verbatim}
		recibir() = (mensaje, marca_temporal_mensaje);
		reloj = max(marca_temporal_mensaje,reloj) + 1;
	\end{verbatim}
	
	\section{Ejercicio de Funcionamiento de relojes de Lamport en procesos}
	
	En un sistema distribuido, existen tres procesos: \textbf{P1}, \textbf{P2} y \textbf{P3}. Estos procesos se comunican mediante el envío y recepción de mensajes. A continuación se muestra una secuencia parcial de eventos (etiquetados con letras) que ocurren en cada proceso:
	
	\begin{itemize}
		\item \textbf{P1}: a → b → c → d
		\item \textbf{P2:} e → f → g
		\item \textbf{P3}: h → i → j → k
	\end{itemize}
	
	Se sabe que se producen los siguientes eventos de comunicación entre procesos:
	
	\begin{itemize}
		\item El evento \textbf{b} envía un mensaje que es recibido por el evento \textbf{f}.
		\item El evento \textbf{h} envía un mensaje que es recibido por el evento \textbf{c}.
		\item El evento \textbf{g} envía un mensaje que es recibido por el evento \textbf{i}.
		\item El evento \textbf{d} envía un mensaje que es recibido por el evento \textbf{k}.
	\end{itemize}
	
	Ejercicio:
	\begin{enumerate}
		\item Asigna marcas de tiempo usando el algoritmo de \textbf{relojes lógicos de Lamport}, suponiendo que:
		\begin{itemize}
			\item Cada proceso incrementa su reloj en 1 unidad entre eventos locales.
			\item El reloj del evento receptor debe ajustarse según el algoritmo de Lamport.
		\end{itemize}
		\item Muestra la tabla con los eventos y sus marcas de tiempo para cada proceso.
	\end{enumerate}
	
	\subsection{Solución}
	
	\begin{itemize}
		\item Cada proceso comienza con reloj = 0.
		\item Para cada evento local: reloj = reloj + 1.
		\item Al enviar un mensaje: se envía la marca de tiempo actual.
		\item Al recibir un mensaje: reloj = max(reloj local, marca del mensaje) + 1.
	\end{itemize}
	

	\section*{Proceso P1 (reloj inicial = 0)}
	
	\begin{table}[h]
		\centering
		\begin{tabular}{|c|c|l|c|}
			\hline
			\textbf{Evento} & \textbf{Tipo} & \textbf{Cálculo} & \textbf{Marca} \\
			\hline
			a & local & $0 + 1 = 1$ & \textbf{1} \\
			\hline
			b & envía a f & $1 + 1 = 2$ (envía 2) & \textbf{2} \\
			\hline
			c & recibe de h (marca 1) & $\max(2, 1) + 1 = 3$ & \textbf{3} \\
			\hline
			d & envía a k & $3 + 1 = 4$ (envía 4) & \textbf{4} \\
			\hline
		\end{tabular}
	\end{table}
	
	\section*{Proceso P2 (reloj inicial = 0)}
	
	\begin{table}[h]
		\centering
		\begin{tabular}{|c|c|l|c|}
			\hline
			\textbf{Evento} & \textbf{Tipo} & \textbf{Cálculo} & \textbf{Marca} \\
			\hline
			e & local & $0 + 1 = 1$ & \textbf{1} \\
			\hline
			f & recibe de b (marca 2) & $\max(1, 2) + 1 = 3$ & \textbf{3} \\
			\hline
			g & envía a i & $3 + 1 = 4$ (envía 4) & \textbf{4} \\
			\hline
		\end{tabular}
	\end{table}
	
	\section*{Proceso P3 (reloj inicial = 0)}
	
	\begin{table}[h]
		\centering
		\begin{tabular}{|c|c|l|c|}
			\hline
			\textbf{Evento} & \textbf{Tipo} & \textbf{Cálculo} & \textbf{Marca} \\
			\hline
			h & envía a c & $0 + 1 = 1$ (envía 1) & \textbf{1} \\
			\hline
			i & recibe de g (marca 4) & $\max(1, 4) + 1 = 5$ & \textbf{5} \\
			\hline
			j & local & $5 + 1 = 6$ & \textbf{6} \\
			\hline
			k & recibe de d (marca 4) & $\max(6, 4) + 1 = 7$ & \textbf{7} \\
			\hline
		\end{tabular}
	\end{table}
	
	\section*{Tabla final consolidada}
	
	\begin{table}[H]
		\centering
		\begin{tabular}{|c|c|c|}
			\hline
			\textbf{Proceso} & \textbf{Evento} & \textbf{Marca Lamport} \\
			\hline
			P1 & a & 1 \\
			\hline
			P1 & b & 2 \\
			\hline
			P1 & c & 3 \\
			\hline
			P1 & d & 4 \\
			\hline
			P2 & e & 1 \\
			\hline
			P2 & f & 3 \\
			\hline
			P2 & g & 4 \\
			\hline
			P3 & h & 1 \\
			\hline
			P3 & i & 5 \\
			\hline
			P3 & j & 6 \\
			\hline
			P3 & k & 7 \\
			\hline
		\end{tabular}
	\end{table}
	
	\section{Conclusión}
	La implementación del algoritmo de Lamport, ejemplificada en el ejercicio con tres procesos, confirma su eficacia para establecer un orden causal lógico robusto en sistemas distribuidos. El mecanismo, que combina incrementos locales y la función de maximización al recibir mensajes, garantiza la coherencia del tiempo lógico ante las asincronías de la comunicación. Los resultados obtenidos validan que este enfoque proporciona una base sólida y sencilla para rastrear relaciones de precedencia, siendo esencial para técnicas avanzadas de sincronización y coordinación en este tipo de entornos.
	
	
	% Bibliografía
	\begin{thebibliography}{1}
		
		\bibitem{WikipediaLamport}
		Wikipedia.
		\newblock {\em Tiempos lógicos de Lamport}.
		\newblock Recuperado de \url{https://es.wikipedia.org/wiki/Tiempos_l%C3%B3gicos_de_Lamport}.
		\newblock Último acceso: \today.
		
	\end{thebibliography}
	
	% The default list of authors is too long for headers}
%\renewcommand{\shortauthors}{G. Zhou et al.}


\end{document}